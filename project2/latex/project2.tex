\documentclass{article}
\usepackage{graphicx}
\usepackage{amsmath}
\usepackage{float}

\title{\textbf{FYS4150/FYS1350 - Project 2}}
\author{Ingvild Bergsbak, Oliver Hebnes and Erlend Ousdal}
\date{October 1}




\begin{document}

\maketitle

\newpage

\section{Abstract}


\section{Introduction}


\section{Theoretical Models and Technicalities}

We have av orthogonal matrix so that
$$\mathbf{v}_j^T\mathbf{v}_i=\delta_{ij}$$
and othogonal transformation $\mathbf{w}$.\\
\vskip0.1cm
$\mathbf{w}_i=\mathbf{Uv}_i$\\
\vskip0.1cm
$\mathbf{w}_j^T\cdot \mathbf{w}_i=(\mathbf{Uv}_j)^T\cdot (\mathbf{Uv}_i)$
\vskip0.5cm
Calculating $\mathbf{Uv}_i$ and $(\mathbf{Uv}_j)^T$ separately.
\begin{equation*}
\mathbf{Uv}_i=\begin{bmatrix}
u_{11} & u_{12} & \cdots & u_{1n}\\
u_{21} & u_{22} & \cdots & u_{2n}\\
\vdots & \vdots & \ddots & \vdots\\
u_{n1} & u_{n2} & \cdots & u_{nn}\\
\end{bmatrix} \begin{bmatrix}
v_{i1} \\
v_{i2} \\
\vdots \\
v_{in} \\
\end{bmatrix}=\begin{bmatrix}
u_{11}v_{i1} + u_{12}v_{i2} + \cdots + u_{1n}v_{in}\\
u_{21}v_{i1} + u_{22}v_{i2} + \cdots + u_{2n}v_{in}\\
\vdots \\
u_{n1}v_{i1} +  u_{n2}v_{i2} + \cdots + u_{nn}v_{in}\\
\end{bmatrix}
\end{equation*}



\begin{equation*}
\begin{split}
(\mathbf{Uv}_j)^T&=\begin{bmatrix}
u_{11}v_{j1} + \cdots + u_{1n}v_{jn} &
u_{21}v_{j1} +  \cdots + u_{2n}v_{jn} &
\cdots &
u_{n1}v_{j1} + \cdots + u_{nn}v_{jn}\\
\end{bmatrix}\\
&=\begin{bmatrix}
v_{j1} &
v_{j2} &
\cdots &
v_{jn} 
\end{bmatrix}\begin{bmatrix}
u_{11} & u_{21} & \cdots & u_{n1}\\
u_{12} & u_{22} & \cdots & u_{n2}\\
\vdots & \vdots & \ddots & \vdots\\
u_{1n} & u_{2n} & \cdots & u_{nn}\\
\end{bmatrix} \\
&=\mathbf{v}_j^T\mathbf{U}^T
\end{split}
\end{equation*}

$\mathbf{Uv}_i$ and $(\mathbf{Uv}_j)^T$ inserted back into the equation gives

\begin{equation*}
\begin{split}
\mathbf{w}_j^T\cdot \mathbf{w}_i&=(\mathbf{Uv}_j)^T\cdot (\mathbf{Uv}_i)\\
&=\mathbf{v}_j^T\mathbf{U}^T\mathbf{Uv}_i\\
&=\mathbf{v}_j^T\mathbf{I}\mathbf{v}_i\\
&=\mathbf{v}_j^T\mathbf{v}_i\\
&=\delta_{ij}
\end{split}
\end{equation*}

The orthogonal transformation preserves orthogonality and the dot product.
\vskip0.5cm

For an eigenvalue problem
$$\mathbf{Ax}=\lambda\mathbf{x}$$
one can insert $\mathbf{B}=\mathbf{S^TAS}$ so that
$$\mathbf{B(S^Tx)}=\lambda(\mathbf{S^Tx})$$
which shows that $\mathbf{A}$ and $\mathbf{B}$ has the same eigenvalues, but different eigenvectors. This means that by finding the eigenvalues for $\mathbf{B}$ we also have the eigenvalues for $\mathbf{A}$.
\vskip0.5cm
Now we want to solve the eigenvalue problem using Jacobi's rotation method. The method uses orthogonal transformations to diagonalize the matrix by plane rotation around an angle $\theta$. The orthogonal transformation matrix is a $n\times n$ matrix on the form

\begin{equation*}
\mathbf{S}=\begin{bmatrix}
1 & 0 & & & \cdots & & & 0\\
0 & 1 & & & & & \\
 &  &\ddots & & & & &\\
\vdots &  & & \cos\theta & \cdots & \sin\theta & &\\
 & & &\vdots & \ddots & & &\\
 & & & -\sin\theta & \cdots & \cos\theta & & \\
 & & & & & & \ddots & \\
0 &  & & & \cdots  & & & 1\\
\end{bmatrix}
\end{equation*}

which has the property $\mathbf{S^T}=\mathbf{S^{-1}}$. The matrix has $1$ along the diagonal except $s_{kk}$ and $s_{ll}$ which is $\cos\theta$. Element $s_{kl}$ and $s_{lk}$ is $-\sin\theta$ and $\sin\theta$ respectively. All other elements are zero.

Performing the similarity transformation 
$$\mathbf{B}=\mathbf{S^TAS}$$
repeatedly will result in a diagonal matrix by choosing an appropriate value for $\theta$ for every transformation.

One transformation has the objective of setting the highest non-diagonal element, $b_{kl}$, to zero and changes the elements $b_{kk}$, $b_{ll}$, $b_{ik}$ and $b_{il}$ for all $i$ except $i=k,l$ the following way

\begin{equation*}
\begin{split}
b_{ik}&=a_{ik}\cos\theta-a_{il}\sin\theta\\
b_{il}&=a_{il}\cos\theta+a_{ik}\sin\theta\\
b_{kk}&=a_{kk}\cos^2\theta-2a_{kl}\cos\theta\sin\theta+a_{ll}\sin^2\theta\\
b_{ll}&=a_{ll}\cos^2\theta+2a_{kl}\cos\theta\sin\theta+a_{kk}\sin^2\theta\\
b_{kl}&=(a_{kk}-a_{ll})\cos\theta\sin\theta+a_{kl}(\cos^2\theta-\sin^2\theta)\\
\end{split}
\end{equation*}

$\theta$ is determined by demanding $b_{kl}=0$.

We perform the similarity transformations until all non-diagonal elements are zero, and we end up with a diagonal matrix. By definition the elements on the diagonal of a diagonal matrix, $\mathbf{D}$ are the eigenvalues of $\mathbf{D}$.

 








\section{Results and Discussion}






























\end{document}



\begin{bmatrix}
1 & 1 & 0 & 2\\
0 & 1 & 0 & 1\\
1 & 0 & 0 & 1\\
1 & 0 & 1 & 2\\
\end{bmatrix}