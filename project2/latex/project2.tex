\documentclass{article}
\usepackage{graphicx}
\usepackage{amsmath}
\usepackage{float}

\title{\textbf{FYS4150/FYS1350 - Project 2}}
\author{Ingvild Bergsbak, Oliver Hebnes and Erlend Ousdal}
\date{October 1}




\begin{document}

\maketitle

\newpage

\section{Abstract}


\section{Introduction}


\section{Theoretical Models and Technicalities}

We have av orthogonal matrix so that
$$\mathbf{v}_j^T\mathbf{v}_i=\delta_{ij}$$
and othogonal transformation $\mathbf{w}$.\\
\vskip0.1cm
$\mathbf{w}_i=\mathbf{Uv}_i$\\
\vskip0.1cm
$\mathbf{w}_j^T\cdot \mathbf{w}_i=(\mathbf{Uv}_j)^T\cdot (\mathbf{Uv}_i)$
\vskip0.5cm
Calculating $\mathbf{Uv}_i$ and $(\mathbf{Uv}_j)^T$ separately.
\begin{equation*}
\mathbf{Uv}_i=\begin{bmatrix}
u_{11} & u_{12} & \cdots & u_{1n}\\
u_{21} & u_{22} & \cdots & u_{2n}\\
\vdots & \vdots & \ddots & \vdots\\
u_{n1} & u_{n2} & \cdots & u_{nn}\\
\end{bmatrix} \begin{bmatrix}
v_{i1} \\
v_{i2} \\
\vdots \\
v_{in} \\
\end{bmatrix}=\begin{bmatrix}
u_{11}v_{i1} + u_{12}v_{i2} + \cdots + u_{1n}v_{in}\\
u_{21}v_{i1} + u_{22}v_{i2} + \cdots + u_{2n}v_{in}\\
\vdots \\
u_{n1}v_{i1} +  u_{n2}v_{i2} + \cdots + u_{nn}v_{in}\\
\end{bmatrix}
\end{equation*}



\begin{equation*}
\begin{split}
(\mathbf{Uv}_j)^T&=\begin{bmatrix}
u_{11}v_{j1} + \cdots + u_{1n}v_{jn} &
u_{21}v_{j1} +  \cdots + u_{2n}v_{jn} &
\cdots &
u_{n1}v_{j1} + \cdots + u_{nn}v_{jn}\\
\end{bmatrix}\\
&=\begin{bmatrix}
v_{j1} &
v_{j2} &
\cdots &
v_{jn} 
\end{bmatrix}\begin{bmatrix}
u_{11} & u_{21} & \cdots & u_{n1}\\
u_{12} & u_{22} & \cdots & u_{n2}\\
\vdots & \vdots & \ddots & \vdots\\
u_{1n} & u_{2n} & \cdots & u_{nn}\\
\end{bmatrix} \\
&=\mathbf{v}_j^T\mathbf{U}^T
\end{split}
\end{equation*}

$\mathbf{Uv}_i$ and $(\mathbf{Uv}_j)^T$ inserted back into the equation gives

\begin{equation*}
\begin{split}
\mathbf{w}_j^T\cdot \mathbf{w}_i&=(\mathbf{Uv}_j)^T\cdot (\mathbf{Uv}_i)\\
&=\mathbf{v}_j^T\mathbf{U}^T\mathbf{Uv}_i\\
&=\mathbf{v}_j^T\mathbf{I}\mathbf{v}_i\\
&=\mathbf{v}_j^T\mathbf{v}_i\\
&=\delta_{ij}
\end{split}
\end{equation*}

The orthogonal transformation preserves orthogonality and the dot product.

Now we want to solve the eigenvalue problem using Jacobi's rotation method. The method uses orthogonal transformations to diagonalize the matrix 















\section{Results and Discussion}






























\end{document}



\begin{bmatrix}
1 & 1 & 0 & 2\\
0 & 1 & 0 & 1\\
1 & 0 & 0 & 1\\
1 & 0 & 1 & 2\\
\end{bmatrix}